\documentclass[12pt]{article}
\usepackage{amsfonts}
\usepackage{fancyhdr}
\usepackage{comment}
\usepackage{amsmath}
\usepackage[a4paper, top=2.5cm, bottom=2.5cm, left=2.2cm, right=2.2cm]%
{geometry}
\usepackage{times}
\usepackage{amsmath}
\usepackage{changepage}
\usepackage{amssymb}
\usepackage{fixltx2e}
\usepackage{enumerate}
\usepackage{graphicx}
\usepackage{float}
\newtheorem{theorem}{Theorem}
\newtheorem{acknowledgement}[theorem]{Acknowledgement}
\newtheorem{algorithm}[theorem]{Algorithm}
\newtheorem{axiom}{Axiom}
\newtheorem{case}[theorem]{Case}
\newtheorem{claim}[theorem]{Claim}
\newtheorem{conclusion}[theorem]{Conclusion}
\newtheorem{condition}[theorem]{Condition}
\newtheorem{conjecture}[theorem]{Conjecture}
\newtheorem{corollary}[theorem]{Corollary}
\newtheorem{criterion}[theorem]{Criterion}
\newtheorem{definition}[theorem]{Definition}
\newtheorem{example}[theorem]{Example}
\newtheorem{exercise}[theorem]{Exercise}
\newtheorem{lemma}[theorem]{Lemma}
\newtheorem{notation}[theorem]{Notation}
\newtheorem{problem}[theorem]{Problem}
\newtheorem{proposition}[theorem]{Proposition}
\newtheorem{remark}[theorem]{Remark}
\newtheorem{solution}[theorem]{Solution}
\newtheorem{summary}[theorem]{Summary}
\newenvironment{proof}[1][Proof]{\textbf{#1.} }{\ \rule{0.5em}{0.5em}}

\newcommand{\Q}{\mathbb{Q}}
\newcommand{\R}{\mathbb{R}}
\newcommand{\C}{\mathbb{C}}
\newcommand{\Z}{\mathbb{Z}}

\begin{document}

\title{SIE 545: Fundamentals of Optimization \\Homework 3}
\author{Martin Manuel Lopez \\lopez9@email.arizona.edu \\Systems and Industrial Engineering}
\date{\today}
\maketitle
\section{BSS 3.1}   
Which of the following functions are convex, concave, or neither? Why?\\ \\
a.) $f(x_1,x_2) = 2x_1^2 - 4x_1x_2 - 8x_1 + 3x_2$\\\\
\begin{align*}
    &\frac{\partial f}{\partial x_1} = 4x_1 - 4x_2 -8\\ \\
    &\frac{\partial f}{\partial x_2} = -4x_1 +3 \\ \\  
        &H(x) = 
        \begin{bmatrix}
          \frac{\partial^2 f}{\partial x_1^2} & \frac{\partial^2 f}{\partial x_1 \partial x_2} \\
          \frac{\partial^2 f}{\partial x_2 \partial x_1} & \frac{\partial^2 f}{\partial x_2^2} \\
        \end{bmatrix} \\ \\
    &H(x) = 
        \begin{bmatrix}
            4 & -4 \\
            -4 & 0 \\
        \end{bmatrix}
\end{align*}\\

Given the H(x) matrix is not Positive Semi-Definite then the function neither convex or concave. \\ \\

b.) $f(x_1,x_2) = x_1 e^{-(x_1 + 3x_2)}$\\\\
\begin{align*}
    &\frac{\partial f}{\partial x_1} = (e^{-3x_2} - x_1e^{-3x_2})e^{-x_1}\\ \\
    &\frac{\partial f}{\partial x_2} = -3x_1e^{-3x_2-x_1} \\ \\  
        &H(x) = 
        \begin{bmatrix}
          \frac{\partial^2 f}{\partial x_1^2} & \frac{\partial^2 f}{\partial x_1 \partial x_2} \\
          \frac{\partial^2 f}{\partial x_2 \partial x_1} & \frac{\partial^2 f}{\partial x_2^2} \\
        \end{bmatrix} \\ \\
    &H(x) = e^{-x_1-3x_2}
        \begin{bmatrix}
            (x_1 - 2) & 3(x_1-1) \\
            3(x_1-1) & 9x_1 \\
        \end{bmatrix}
\end{align*}\\

Given the $H(x)$ matrix is not Positive Semi-Definite then the function neither convex or concave because the $H(X)$ is dependent on $x_1$.\\

c.) $f(x_1,x_2) = -x_1^2 -3x_2^2 + 4x_1x_2 + 10x_1 - 10x_2$\\\\
\begin{align*}
    &\frac{\partial f}{\partial x_1} = -2x_1 + 4x_2 + 10\\ \\
    &\frac{\partial f}{\partial x_2} = -6x_2 + 4x_1 - 10 \\ \\  
        &H(x) = 
        \begin{bmatrix}
          \frac{\partial^2 f}{\partial x_1^2} & \frac{\partial^2 f}{\partial x_1 \partial x_2} \\
          \frac{\partial^2 f}{\partial x_2 \partial x_1} & \frac{\partial^2 f}{\partial x_2^2} \\
        \end{bmatrix} \\ \\
    &H(x) =
        \begin{bmatrix}
            -2 & 4 \\
            4 & -6 \\
        \end{bmatrix}\\ \\
    &-H(x) =
        \begin{bmatrix}
            2 & -4 \\
            -4 & 6 \\
        \end{bmatrix}\\ \\
    &det(H(x)) = 12-16 = -4 < 0\\ \\
    &det(-H(x)) = 12 -16 = -4 < 0 \\ \\
\end{align*}\\

Given the $H(x)$ matrix is not Positive Semi-Definite then the function neither convex or concave. Evaluating $-H(x)$ we still determine that the negative of $H(x)$ is not PSD so it is not concave. The function is neither convex or concave.\\

c.) $f(x_1,x_2,x_3) = 2x_1x_2 + 2x_1^2 + x_2^2 + 2x_3^2 -5x_1x_3$\\\\
\begin{align*}
    &\frac{\partial f}{\partial x_1} = 2x_2 + 4x_1 - 5x_3\\ \\
    &\frac{\partial f}{\partial x_2} = 2x_1 + 2x_2 \\ \\ 
    &\frac{\partial f}{\partial x_3} = 4x_3 - 5x_1 \\ \\
    &H(x) = 
        \begin{bmatrix}
            4 & 2 & -5 \\
            2 & 2 & 0 \\
            -5 & 0 & 4 \\
        \end{bmatrix}\\
    & 2R_2 - R_1 = R_2 \\   
    &H(x) = 
        \begin{bmatrix}
            4 & 2 & -5 \\
            0 & 2 & 5 \\
            -5 & 0 & 4 \\
        \end{bmatrix}\\
    &5R_1 + 4R_3 = R_3 \\
    &H(x) = 
        \begin{bmatrix}
            4 & 2 & -5 \\
            0 & 2 & 5 \\
            0 & 10 & -9 \\
        \end{bmatrix}\\
    &\text{We take the following matrix based on Gauss-Jordan method:} 
    &H(x) =
        \begin{bmatrix}
            2 & 5 \\
            10 & -9 \\
        \end{bmatrix}\\ \\
    &(2 * -9) -(10 *5) = -68 < 0\\ \\
        &-H(x) =
        \begin{bmatrix}
            -2 & -5 \\
            -10 & 9 \\
        \end{bmatrix}\\ \\
    &(-2 * 9) -(-10 *-5) = -68 < 0\\
\end{align*}\\

Given the $H(x)$ matrix is not Positive Semi-Definite then the function neither convex or concave.The $-H(x)$ is not PSD and thus it is not concave. \\ \\
\section{BSS 3.10} 
Let h: $\R^n$ \longrightarrow $\R$ \text{be a concave function, and let} g: $\R$ \longrightarrow $\R$ \text{be a non-decreasing convex function. Consider composite function} f: $\R^n$ \longrightarrow \R \text{defined by } $f(x) = g[h(x)]$. \text{Show that $f$ is convex.}\\\\
\text{Given } $f$: S \longrightarrow $\R$ \text{, where S}\subseteq $\R^n$ \text{ is a non empty convex set $f$ is convex on S if 2 points $(x_1, x_2)$.}\\
        $f(\lambda x_1 + (1-\lambda) x_2) \leq \lambda f(x_1) + (1-\lambda) f(x_2)$\\
        let $\lambda$ = 1/2 \\ 
        $f(1/2 x_1 + 1/2 x_2) \leq 1/2 f(x_1) + 1/2 (fx_2)$\\
        Thus\\
        $f(\lambda x_1 + (1-\lambda) x_2) \leq \lambda f(x_1) + (1-\lambda) f(x_2) \forall x_1, x_2 \in S, \forall \lambda \in (0,1). $ \\ 

Since h is a convex function thus we let $x_1 and x_2$ in $h(x)$ be:\\
$h(lambda x_1 + (1-\lambda) x_2) \leq \lambda h(x_1) + (1-\lambda) h(x_2)$\\ 
g : is a non-decreasing convex function thus 2 point $g(x_1)$ and $g(x_2)$ we show the convexity such that we have a convex combination:\\\\
$g(\lambda x_1 + (1-\lambda)x_2) \leq \lambda g(x_1) + (1-\lambda) g(x_2)$\\ \\
Since $f$ is convex and $f(x) = g[h(x)]$ , $f$ is evaluated at any convex combination of 2 points and shall be no longer than the same convex combination of $f(g[h(x_1)])$ and $f(g[h(x_2)])$. \\ \\ 
$f(lambda g[h(x_1)] + (1-\lambda) g[h(x_2)]) \leq \lambda f(g[h(x_1)]) + (1-\lambda) f(g[h(x_2)])$\\ 
\Leftrightarrow \\ 
$f(x) = g[(\lambda h(x_1) + (1-\lambda) h(x_2) \leq \lambda g[h(x_1)] + (1-\lambda) g[h(x_2)]$\\ 
Since h: $\R^n$ \longrightarrow \R \text{and g is non-decreasing convex function we have proved that }$f(x) = g[h(x)]$ \text{is a convex as well.}\\
\section{BSS 3.11} 
Let g: $\R^n$ \longrightarrow $\R$ \text{ be a convex function, and let $f$ be defined by } f(x) = \frac{1}{g(x)}. \text{Show that $f$ is convex over } S = [x: g(x) > 0]. \text{State symmetric result interchanging the convex and concave functions.}\\

    \begin{align*}
        &\forall x_1, x_2 \in \text{S and } \lambda \in (0,1)\\ \\
        &f(\lambda x_1 + (1-\lamda) x_2) \leq \lambda f(x_1) + (1-\lambda) x_2\\ \\ 
        &\text{Based on the definition $f(x)$ is convex}\\ \\
        &\Leftrightarrow\\
        &f(\lambda x_1 + (1-\lambda)x_2) = \frac{1}{g(\lambda x_1 + (1-\lambda) x_2)}\\
        &\lambda f(x_1) + (1-\lambda) f(x_2) = \frac{\lambda}{g(x_1)} + \frac{(1-\lambda)}{g(x_2)}\\
        &\Leftrightarrow\\
        &\frac{\lambda}{g(x_1)} + \frac{1-\lambda}{g(x_2)} - \frac{1}{g(\lambda x_1 + (1-\lambda) x_2)} \geq 0 \\ \\
        &\text{We see that } g(x) \text{is concave} \\
        &\Leftrightarrow\\
        &g(\lambda x_1 + ( 1-\lambda)x_2) \geq \lambda g(x_1) + (1-\lambda) g(x_2) \\ 
        &\Rightarrow\\
        &\frac{1}{g(\lambda x_1 + (1-\lambda) x_2} \leq \frac{1}{\lambda g(x_1) + (1-\lambda) g(x_2)}\\
        &\Rightarrow\\
        &0 \geq \frac{\lambda}{g(x_1)} + \frac{1-\lambda}{g(x_2)} - \frac{1}{g(\lambda x_1 + (1-\lambda) x_2)}\\
        &\Rightarrow\\
        &\frac{\lambda g(x_1) (\lambda g(x_1) + (1-\lambda) g(x_2) + (1-\lambda) g(x_1) (\lambda g(x_1) + (1-\lambda) g(x_2)) - g(x_1)g(x_2))}{g(x_1)g(x_2) (\lambda g(x_1) + (1-\lambda) g(x_2))}\\
        &\Rightarrow\\
        &\frac{\lambda (1-\lambda)(g(x_1)-g(x_2))^2}{g(x_1)g(x_2)(\lambda g(x_1) + (1-\lambda) g(x_2))} \geq 0\\
        &\forall x_1, x_2 \in S, \forall \lambda \in (0,1)\\
        &\text{If g is concave over S, then } \frac{-1}{g} \text{is convex over S and } f(x) = \frac{1}{g(x)} \text{is convex}.
    \end{align*}\\
\section{BSS 3.18}
A function $f: \R^n \longrightarrow \R$ calleda gauge function if it satisfies the following condition: 
    \begin{align*}
        &f(\lambda x) = \lambda f(x) \quad\forall x \in \R , \forall \lambda \geq 0 \\
    \end{align*}
Further, a gauge function is said to be sub-additive of it satisfies the following inequality:
    \begin{align*}
        &f(x) + f(y) \geq f(x+y) \quad \forall x,y \in \R^n \\ 
    \end{align*}
Prove that sub-additivity is equivalent to convexity of the following guage functions: 
    \begin{align*}
        &f(\lambda x) = \lambda f(x) \quad \forall x \in \R^n, \forall\lambda \geq 0\\
    \end{align*}
Relaxing the $f(\lambda x)$ gauge function we assume convexity given two points $x_1, x_2$ evaluated at any convex combination. 
    \begin{align*}
        &f(\lambda x_1 + (1-\lambda) x_2) \leq \lambda f(x_1) + (1-\lambda) f(x_2) \quad \forall x_1, x_2 \in S, \lambda \in (0,1) \\ 
        &\text{Let } x_2 = y \text{and } x_1 = x \\ 
        &\Leftrightarrow\\
        &f(\lambda x + (1-\lambda) y) \leq \lambda f(x) + (1-\lambda) f(y) \quad \forall x,y \in S, \lambda \in (0,1)\\
        &\Leftrightarrow\\
        &f(x+y) \leq f(\lambda x + (1-\lambda) y) \leq \lambda f(x) + (1-\lambda) f(y) \leq f(x) + f(y) \quad \forall x,y \in S, \lambda \in (0,1)\\
        &\Leftrightarrow\\
        &f(x+y) \leq f(x) + f(y) \quad \forall x,y \in \R^n
    \end{align*}
Thus sub-additivity is equivalent to convex gauge functions. 
\section{BSS 3.30}
Consider the function $\theta$ defined by the following optimization problem: 
    \begin{align*}
        &\thata (u_1, u_2) = \min x_2(2-u_1) + x_2 (3-u_2)
        &s.t.
        &x_1^2 + x_2^2 \leq 4
    \end{align*}
a.) Show $\theta$ is concave
    \begin{align*}
        &\theta:\R^n \longrightarrow \R \text{and is a non empty convex set in } \R^n. \Theta \text{is continuous and our assumption is the } \forall x \in S \text{is compact.}\\ 
        &\theta (\lambda u_1 + (1-\lambda) u_2) \geq \lambda \theta(u_1) + (1-\lambda) \theta(u_2) \quad \forall u \in S, \lambda \in (0,1)\\
    \end{align*}\\
b.) $\theta$ (2,3)\\
    \begin{align*}
        &\theta (2,3) = \min x_1(2-u_1) + x_2 (3 - u_2) 
        &\Leftrightarrow\\
        &\theta (2,3) = \min x_1(2-2) + x_2 (3 - 3)\\
        &\theta (2,3) = 0 \\
    \end{align*}
c.) Find collection of subgradients of $\theta$ at (2,3)\\ 
    \begin{align*}
         &\theta (\lambda u_1 + (1-\lambda) u_2) \geq \lambda \theta(u_1) + (1-\lambda) \theta(u_2) \\ 
         &\Leftrightarrow\\
         &\theta (u) \leq \theta (\bar u) + \xi^T (u -\bar u) \quad \text{where } \bar u = (2,3)\\
         &\Leftrightarrow\\
         &\theta(u) = \min x_1 (-u_1 +2) + x_2 (-u_2 + 3) \\ 
         &\Leftrightarrow\\
         & \min -x_1(u_1 -2) -x_2 (u_2 -3) 
         &s.t.\\
         &x_1^2 + x_2^2 \leq 4\\
         &\Leftrightarrow\\
         & \min -\xi_1(u_1 -2) -\xi_2 (u_2 -3) 
         &s.t.\\
         &\xi_1^2 + \xi_2^2 \leq 4\\
         &\Leftrightarrow\\
         & \min -\xi_1(u_1 -2) -\xi_2 (u_2 -3) \leq \min \xi_1(u_1 -2) + \xi_2 (u_2 -3)\\
         &\Leftrightarrow\\
         &\min \xi_1(u_1 -2) + \xi_2 (u_2 -3)\\
         &s.t.\\
         &\xi_1^2 + \xi_2^2 \leq 4\\
    \end{align*}
\section{BSS 3.37}
Let $f: \R^n \longrightarrow \R$ be a differentiable function. The linear approximation of the $f$ at a given point \bar x is given by:\\ 
    \begin{align*}
        &f(\bar x) + \nabla f(\bar x)^t (x-\bar x)\\ 
    \end{align*}
If $f$ is twice differentiable at \bar x, the quadratic approximation of $f$ at \bar x is given by:\\
    \begin{align*}
        &f(\bar x) + \nabla f(\bar x)^t (x-\bar x) + 1/2 (x-\bar x)^t H(\bar x) (x-\bar x)\\ 
    \end{align*}
Let $f(x_1, x_2) = e^{2x_1^2 - x_2^2} - 3x_1 + 5x_2$ given the linear and quadratic approximations of $f$ at (1,1). Are these approximations convex, concave, or neither? Why?\\ 
    \begin{align*}
        &\text{Linear Approximation of }f(1,1): \\
        &f(\bar x) + \nabla f(\bar x)^t (x-\bar x)\\
        &\Leftrightarrow\\
        &\frac{\partial f}{\partial x_1 } = 4x_1 e^{2x_1^2 -x_2^2} - 3\\
        &\frac{\partial f}{\partial x_2 } = 5 -2e^{2x_1^2 -x_2^2}\\
        &\Leftrightarrow\\
        &f(\bar x) + \nabla f(\bar x)^t (x-\bar x) = (2+e)+ (x_1 -1) (4e -3) + (x_2 -1 ) (5-2e)\\
    \end{align*}
We see that the first order Taylor Series expansion is showed in the linear approximation and see that the function is both convex and concave because it is an affine combination.\\
    \begin{align*}
        &\text{Quadritic Approximation of } f(1,1): \\ 
        &f(\bar x) + \nabla f(\bar x)^t (x-\bar x) + 1/2 (x-\bar x)^t H(\bar x) (x-\bar x)\\ 
        &H(x) = 
        \begin{bmatrix}
          \frac{\partial^2 f}{\partial x_1^2} & \frac{\partial^2 f}{\partial x_1 \partial x_2} \\
          \frac{\partial^2 f}{\partial x_2 \partial x_1} & \frac{\partial^2 f}{\partial x_2^2} \\
        \end{bmatrix} \\
        &H(x) = 2e^{2x_1^2 -x_2^2}
        \begin{bmatrix}
            8x_1 + 2 & -4x_1x_2\\
            -4x_1x_2 & 2x_2^2-1\\
        \end{bmatrix} \\ \\
         &H(1,1) = 2e
        \begin{bmatrix}
            10 & -4\\
            -4 & 1\\
        \end{bmatrix} \\ \\
         &-H(1,1) = 2e
        \begin{bmatrix}
            -10 & 4\\
            4 & -1\\
        \end{bmatrix} \\ \\
        & det(H(1,1)) < 0\\
        & det(-H(1,1)) < 0 \\
    \end{align*}
The quadratic approximation is neither convex or concave as H(1,1) is not PSD and -H(1,1) is also not PSD the function neither convex or concave.
\end{document}